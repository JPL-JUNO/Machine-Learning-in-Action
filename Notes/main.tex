\documentclass[openany]{book}
\include{preamble.tex}

\begin{document}

\maketitle
\tableofcontents
\part{分类}
[前两部分主要探讨监督学习(supervised learning)。在监督学习的过程中,我们只需要给定输入样本集,机器就可以从中推演出指定目标变量的可能结果。

监督学习一般使用两种类型的目标变量 :标称型和数值型。标称型目标变量的结果只在有限目标集中取值,如真与假、动物分类集合 \{ 爬行类、鱼类、哺乳类、两栖类 \} ;数值型目标变量则可以从无限的数值集合中取值,如 0.100、42.001、1000.743 等。数值型目标变量主要用于回归分析。]
% \chapter{机器学习基础}

\include{ch02.tex}
% \include{ch03.tex}
% \include{ch04.tex}
% \include{ch05.tex}
% \include{ch06.tex}
% \include{ch07.tex}
% \include{ch08.tex}
% \include{ch09.tex}
% \include{ch10.tex}
\end{document}