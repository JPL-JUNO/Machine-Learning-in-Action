\chapter{基于概率论的分类方法:\\
  朴素贝叶斯\label{chapter04}}
先统计特征在数据集中取某个特定值的次数,然后除以数据集的实例总数,就得到了特征取该值的概率。我们将在此基础上深入讨论。

从一个最简单的概率分类器开始,然后给出一些假设来学习朴素贝叶斯分类器。我们称之为“朴素”,是因为整个形式化过程只做最原始、最简单的假设。
\section{基于贝叶斯决策理论的分类方法}
\begin{tcolorbox}[title=朴素贝叶斯]
    优点:在数据较少的情况下仍然有效,可以处理多类别问题。\\
    缺点:对于输入数据的准备方式较为敏感。\\
    适用数据类型:标称型数据。
\end{tcolorbox}


\figures{Two probability distribution with known parameters describing the distribution}

假设现在我们有一个数据集,它由两类数据组成,数据分布\autoref{Two probability distribution with known parameters describing the distribution}所示。

假设有位读者找到了描述图中两类数据的统计参数。(暂且不用管如何找到描述这类数据的统计参数,\autoref{chapter10}会详细介绍。)我们现在用$p1(x,y)$表示数据点$(x,y)$属于类别1(图中用圆点表示的类别)的概率,用$p2(x,y)$表示数据点$(x,y)$属于类别2(图中用三角形表示的类别)的概率,那么对于一个新数据点$(x,y)$,可以用下面的规则来判断它的类别:
\begin{itemize}
    \item 如果 $p1(x,y) > p2(x,y)$,那么类别为1。
    \item 如果 $p2(x,y) > p1(x,y)$,那么类别为1。
\end{itemize}
也就是说,我们会选择高概率对应的类别。这就是贝叶斯决策理论的核心思想,即\important{选择具有最高概率的决策}(朴素贝叶斯是贝叶斯决策理论的一部分)。

\begin{tcolorbox}[title=贝叶斯?]
    这里使用的概率解释属于贝叶斯概率理论的范畴,该理论非常流行且效果良好。贝叶斯概率先验知识和逻辑推理来处理不确定命题。另一种概率解释称为频数概率(frequency probability),它只从数据本身获得结论,并不考虑逻辑推理及先验知识。
\end{tcolorbox}