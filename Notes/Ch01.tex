\chapter{机器学习基础}
机器学习能让我们自数据集中受到启发,换句话说,我们会利用计算机来彰显数据背后的真实含义,这才是机器学习的真实含义。它既不是只会徒然模仿的机器人,也不是具有人类感情的仿生人。
\section{何谓机器学习}
机器学习就是把无序的数据转换成有用的信息。(看似无序的)

为什么需要统计学?现实世界中存在着很多例子,我们无法为之建立精确的数学模型,而为了解决这类问题,我们就需要统计学工具。(世界是概率的)
\subsection{机器学习非常重要}
\begin{quotation}
    “我不断地告诉大家,未来十年最热门的职业是统计学家。很多人认为我是开玩笑,谁又能想到计算机工程师会是20世纪90年代最诱人的职业呢?如何解释数据、处理数据、从中抽取价值、展示和交流数据结果,在未来十年将是最重要的职业技能,甚至是大学,中学,小学的学生也必需具备的技能,因为我们每时每刻都在接触大量的免费信息,如何理解数据、从中抽取有价值的信息才是其中的关键。这里统计学家只是其中的一个关键环节,我们还需要合理的展示数据、交流和利用数据。我确实认为,能够从数据分析中领悟到有价值信息是非常重要的。职业经理人尤其需要能够合理使用和理解自己部门产生的数据。”
    \begin{flushright}
        ——McKinsey Quarterly,2009年1月
    \end{flushright}
\end{quotation}
\section{关键术语}
在开始研究机器学习算法之前,必须掌握一些基本的术语。通过构建下面的鸟类分类系统,我们将接触机器学习涉及的常用术语。这类系统非常有趣,通常与机器学习中的专家系统有关。

\autoref{1-1}是我们用于区分不同鸟类需要使用的四个不同的属性值,我们选用体重、翼展、有无脚蹼以及后背颜色作为评测基准。下面测量的这四种值称之为特征,也可以称作属性,但这里一律将其称为特征。\autoref{1-1}中的每一行都是一个具有相关特征的实例。

\begin{table}
    \centering
    \caption{基于四种特征的鸟物种分类表}
    \label{1-1}
    \begin{tabular}{crrccl}
        \hline
          & 体重(克)  & 翼展(厘米) & 脚 蹼 & 后背颜色 & 种 属    \\
        \hline
        1 & 1000.1 & 125.0  & 无   & 棕色   & 红尾鵟    \\
        2 & 3000.7 & 200.0  & 无   & 灰色   & 鹭鹰     \\
        3 & 3300.0 & 220.3  & 无   & 灰色   & 鹭鹰     \\
        4 & 4100.0 & 136.0  & 有   & 黑色   & 普通潜鸟   \\
        5 & 3.0    & 11.0   & 无   & 绿色   & 瑰丽蜂鸟   \\
        6 & 570.0  & 75.0   & 无   & 黑色   & 象牙喙啄木鸟 \\
        \hline
    \end{tabular}
\end{table}

\autoref{1-1}的前两种特征是数值型,可以使用十进制数字;第三种特征(是否有脚蹼)是二值型,只可以取0或1;第四种特征(后背颜色)是基于自定义调色板的枚举类型,这里仅选择一些常用色彩。

假定我们可以得到所需的全部特征信息,那该如何判断鸟是不是象牙喙啄木鸟呢?这个任务就是分类,有很多机器学习算法非常善于分类。

最终我们决定使用某个机器学习算法进行分类,首先需要做的是算法训练,即学习如何分类。通常我们为算法输入大量已分类数据作为算法的训练集。训练集是用于训练机器学习算法的数据样本集合,\autoref{1-1}是包含六个训练样本的训练集,每个训练样本有4种特征、一个目标变量。目标变量是机器学习算法的预测结果,在分类算法中目标变量的类型通常是标称型的,而在回归算法中通常是连续型的。我们通常将分类问题中的目标变量称为类别,并假定分类问题只存在有限个数的类别。
\begin{tcolorbox}
    特征或者属性通常是训练样本集的列,它们是独立测量得到的结果,多个特征联系在一起共同组成一个训练样本。
\end{tcolorbox}
为了测试机器学习算法的效果,通常使用两套独立的样本集:训练数据和测试数据。当机器学习程序开始运行时,使用训练样本集作为算法的输入,训练完成之后输入测试样本。输入测试样本时并不提供测试样本的目标变量,由程序决定样本属于哪个类别。比较测试样本预测的目标变量值与实际样本类别之间的差别,就可以得出算法的实际精确度。

假定这个鸟类分类程序,经过测试满足精确度要求,是否我们就可以看到机器已经学会了如何区分不同的鸟类了呢?这部分工作称之为知识表示,某些算法可以产生很容易理解的知识表示,而某些算法的知识表示也许只能为计算机所理解。知识表示可以采用规则集的形式,也可以采用概率分布的形式,甚至可以是训练样本集中的一个实例。在某些场合中,人们可能并不想建立一个专家系统,而仅仅对机器学习算法获取的信息感兴趣。此时,采用何种方式表示知识就显得非常重要了。
\section{机器学习的主要任务}
已经介绍了机器学习如何解决分类问题,它的主要任务是将实例数据划分到合适的分类中。机器学习的另一项任务是回归,它主要用于预测数值型数据。分类和回归属于监督学习,之所以称之为监督学习,是因为这类算法必须知道预测什么,即目标变量的信息。

与监督学习相对应的是无监督学习,此时数据没有类别信息,也不会给定目标值。在无监督学习中,将数据集合分成由类似的对象组成的多个类的过程被称为聚类;将寻找描述数据统计值的过程称之为密度估计。此外,无监督学习还可以减少数据特征的维度,以便我们可以使用二维或三维图形更加直观地展示数据信息。

\begin{itemize}
    \item 监督学习:k-近邻算法、线性回归、朴素贝叶斯算法、局部加权线性回归、支持向量机、Ridge 回归、决策树、Lasso 最小回归系数估计
    \item 无监督学习:K-均值、最大期望算法、DBSCAN、Parzen窗设计
\end{itemize}
\section{如何选择合适的算法}
选择实际可用的算法,必须考虑下面两个问题:
\begin{enumerate}
    \item 使用机器学习算法的目的,想要算法完成何种任务;
    \item 需要分析或收集的数据是什么;
\end{enumerate}

首先考虑使用机器学习算法的目的。如果想要预测目标变量的值,则可以选择监督学习算法,否则可以选择无监督学习算法。确定选择监督学习算法之后,需要进一步确定目标变量类型,如果目标变量是离散型,则可以选择分类器算法;如果目标变量是连续型的数值,则需要选择回归算法。如果不想预测目标变量的值,则可以选择无监督学习算法。进一步分析是否需要将数据划分为离散的组。如果这是唯一的需求,则使用聚类算法;如果还需要估计数据与每个分组的相似程度,则需要使用密度估计算法。

其次需要考虑的是数据问题。我们应该充分了解数据,对实际数据了解得越充分,越容易创建符合实际需求的应用程序。主要应该了解数据的以下特性:特征值是离散型变量还是连续型变量,特征值中是否存在缺失的值,何种原因造成缺失值,数据中是否存在异常值,某个特征发生的频率如何(是否罕见得如同海底捞针),等等。充分了解上面提到的这些数据特性可以缩短选择机器学习算法的时间。

我们只能在一定程度上缩小算法的选择范围,一般并不存在最好的算法或者可以给出最好结果的算法,同时还要尝试不同算法的执行效果。对于所选的每种算法,都可以使用其他的机器学习技术来改进其性能。在处理输入数据之后,两个算法的相对性能也可能会发生变化。一般说来发现最好算法的关键环节是反复试错的迭代过程。
\section{开发机器学习应用程序的步骤}
机器学习算法虽然各不相同,但是使用算法创建应用程序的步骤却基本类似。

学习和使用机器学习算法开发应用程序,通常遵循以下的步骤。

\begin{enumerate}
    \item 收集数据。我们可以使用很多方法收集样本数据,如:制作网络爬虫从网站上抽取数据、从RSS反馈或者API中得到信息、设备发送过来的实测数据(风速、血糖等)。
    \item  准备输入数据。得到数据之后,还必须确保数据格式符合要求,本书采用的格式是Python语言的List。使用这种标准数据格式可以融合算法和数据源,方便匹配操作。

          此外还需要为机器学习算法准备特定的数据格式,如某些算法要求特征值使用特定的格式,一些算法要求目标变量和特征值是字符串类型,而另一些算法则可能要求是整数类型。
    \item 分析输入数据。此步骤主要是人工分析以前得到的数据。这一步的主要作用是确保数据集中没有垃圾数据。如果是在产品化系统中使用机器学习算法并且算法可以处理系统产生的数据格式,或者我们信任数据来源,可以直接跳过第3步。此步骤
          需要人工干预,如果在自动化系统中还需要人工干预,显然就降低了系统的价值。
    \item 训练算法。我们将前两步得到的格式化数据输入到算法,从中抽取知识或信息。
    \item  测试算法。对于监督学习,必须已知用于评估算法的目标变量值;对于无监督学习,也必须用其他的评测手段来检验算法的成功率。无论哪种情形,如果不满意算法的输出结果,则可以回到第4步,改正并加以测试。问题常常会跟数据的收集和准备有关,这时你就必须跳回第1步重新开始。
    \item 使用算法。将机器学习算法转换为应用程序,执行实际任务,以检验上述步骤是否可以在实际环境中正常工作。
\end{enumerate}

\section{Python 语言的优势}
基于以下三个原因,我们选择Python作为实现机器学习算法的编程语言:(1) Python的语法清晰;(2) 易于操作纯文本文件;(3) 使用广泛,存在大量的开发文档。

\subsection{可执行伪代码}
Python具有清晰的语法结构,大家也把它称作可执行伪代码(executable pseudo-code)。

Python语言处理和操作文本文件非常简单,非常易于处理非数值型数据。Python语言提供了丰富的正则表达式函数以及很多访问Web页面的函数库,使得从HTML中提取数据变得非常简单直观。

\subsection{Python 比较流行}
Python语言使用广泛,代码范例也很多,便于读者快速学习和掌握。此外,在开发实际应用程序时,也可以利用丰富的模块库缩短开发周期。

\subsection{Python 语言的特色}
Python语言则与Java和C完全不同,它清晰简练,而且易于理解,即使不是编程人员也能够理解程序的含义,而Java和C对于非编程人员则像天书一样难于理解。

\subsection{Python 语言的缺点}
Python语言唯一的不足是性能问题。Python程序运行的效率不如Java或者C代码高,但是我们可以使用Python调用C编译的代码。这样,我们就可以同时利用C和Python的优点,逐步地开发机器学习应用程序。

C++ Boost库就适合完成这个任务,其他类似于Cython和PyPy的工具也可以编写强类型的Python代码,改进一般Python程序的性能。

如果程序的算法或者思想有缺陷,则无论程序的性能如何,都无法得到正确的结果。如果解决问题的思想存在问题,那么单纯通过提高程序的运行效率,扩展用户规模都无法解决这个核心问题。从这个角度来看,Python快速实现系统的优势就更加明显了,我们可以快速地检验算法或者思想是否正确,如果需要,再进一步优化代码。
\section{Numpy函数库基础}
开始学习机器学习算法之前,必须确保可以正确运行Python开发环境,同时正确安装了NumPy函数库。



这里有代码可以简单的了解Numpy的运行,\href{https://github.com/JPL-JUNO/Machine-Learning-in-Action/blob/main/Codes/Ch01.ipynb}{点击这里}。

\begin{tcolorbox}[title=NumPy矩阵与数组的区别]
    NumPy函数库中存在两种不同的数据类型(矩阵matrix和数组array),都可以用于处理行列表示的数字元素。虽然它们看起来很相似,但是在这两个数据类型上执行相同的数学运算可能得到不同的结果,其中NumPy函数库中的matrix与MATLAB中matrices等价。
\end{tcolorbox}

\section{本章小结}
每天我们需要处理的数据在不断地增加,能够深入理解数据背后的真实含义,是数据驱动产业必须具备的基本技能。

尽管我们构造的专家系统无法像人类专家一样精确地识别或者预测,然而构建接近专家水平的机器系统可以显著地改进我们的生活质量。