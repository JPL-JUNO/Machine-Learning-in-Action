\chapter{机器学习基础}
机器学习能让我们自数据集中受到启发,换句话说,我们会利用计算机来彰显数据背后的真实含义,这才是机器学习的真实含义。它既不是只会徒然模仿的机器人,也不是具有人类感情的仿生人。
\section{何谓机器学习}
机器学习就是把无序的数据转换成有用的信息。(看似无序的)

为什么需要统计学?现实世界中存在着很多例子,我们无法为之建立精确的数学模型,而为了解决这类问题,我们就需要统计学工具。(世界是概率的)
\subsection{机器学习非常重要}
\begin{quotation}
    “我不断地告诉大家,未来十年最热门的职业是统计学家。很多人认为我是开玩笑,谁又能想到计算机工程师会是20世纪90年代最诱人的职业呢?如何解释数据、处理数据、从中抽取价值、展示和交流数据结果,在未来十年将是最重要的职业技能,甚至是大学,中学,小学的学生也必需具备的技能,因为我们每时每刻都在接触大量的免费信息,如何理解数据、从中抽取有价值的信息才是其中的关键。这里统计学家只是其中的一个关键环节,我们还需要合理的展示数据、交流和利用数据。我确实认为,能够从数据分析中领悟到有价值信息是非常重要的。职业经理人尤其需要能够合理使用和理解自己部门产生的数据。”
    \begin{flushright}
        ——McKinsey Quarterly,2009年1月
    \end{flushright}
\end{quotation}
\section{关键术语}
在开始研究机器学习算法之前,必须掌握一些基本的术语。通过构建下面的鸟类分类系统,我们将接触机器学习涉及的常用术语。这类系统非常有趣,通常与机器学习中的专家系统有关。

\autoref{1-1}是我们用于区分不同鸟类需要使用的四个不同的属性值,我们选用体重、翼展、有无脚蹼以及后背颜色作为评测基准。下面测量的这四种值称之为特征,也可以称作属性,但这里一律将其称为特征。\autoref{1-1}中的每一行都是一个具有相关特征的实例。

\begin{table}
    \centering
    \caption{基于四种特征的鸟物种分类表}
    \label{1-1}
    \begin{tabular}{crrccl}
        \hline
          & 体重(克)  & 翼展(厘米) & 脚 蹼 & 后背颜色 & 种 属    \\
        \hline
        1 & 1000.1 & 125.0  & 无   & 棕色   & 红尾鵟    \\
        2 & 3000.7 & 200.0  & 无   & 灰色   & 鹭鹰     \\
        3 & 3300.0 & 220.3  & 无   & 灰色   & 鹭鹰     \\
        4 & 4100.0 & 136.0  & 有   & 黑色   & 普通潜鸟   \\
        5 & 3.0    & 11.0   & 无   & 绿色   & 瑰丽蜂鸟   \\
        6 & 570.0  & 75.0   & 无   & 黑色   & 象牙喙啄木鸟 \\
        \hline
    \end{tabular}
\end{table}

\autoref{1-1}的前两种特征是数值型,可以使用十进制数字;第三种特征(是否有脚蹼)是二值型,只可以取0或1;第四种特征(后背颜色)是基于自定义调色板的枚举类型,这里仅选择一些常用色彩。

假定我们可以得到所需的全部特征信息,那该如何判断鸟是不是象牙喙啄木鸟呢?这个任务就是分类,有很多机器学习算法非常善于分类。

最终我们决定使用某个机器学习算法进行分类,首先需要做的是算法训练,即学习如何分类。通常我们为算法输入大量已分类数据作为算法的训练集。训练集是用于训练机器学习算法的数据样本集合,\autoref{1-1}是包含六个训练样本的训练集,每个训练样本有4种特征、一个目标变量。目标变量是机器学习算法的预测结果,在分类算法中目标变量的类型通常是标称型的,而在回归算法中通常是连续型的。我们通常将分类问题中的目标变量称为类别,并假定分类问题只存在有限个数的类别。
\begin{tcolorbox}
    特征或者属性通常是训练样本集的列,它们是独立测量得到的结果,多个特征联系在一起共同组成一个训练样本。
\end{tcolorbox}
为了测试机器学习算法的效果,通常使用两套独立的样本集:训练数据和测试数据。当机器学习程序开始运行时,使用训练样本集作为算法的输入,训练完成之后输入测试样本。输入测试样本时并不提供测试样本的目标变量,由程序决定样本属于哪个类别。比较测试样本预测的目标变量值与实际样本类别之间的差别,就可以得出算法的实际精确度。

假定这个鸟类分类程序,经过测试满足精确度要求,是否我们就可以看到机器已经学会了如何区分不同的鸟类了呢?这部分工作称之为知识表示,某些算法可以产生很容易理解的知识表示,而某些算法的知识表示也许只能为计算机所理解。知识表示可以采用规则集的形式,也可以采用概率分布的形式,甚至可以是训练样本集中的一个实例。在某些场合中,人们可能并不想建立一个专家系统,而仅仅对机器学习算法获取的信息感兴趣。此时,采用何种方式表示知识就显得非常重要了。
\section{机器学习的主要任务}
已经介绍了机器学习如何解决分类问题,它的主要任务是将实例数据划分到合适的分类中。机器学习的另一项任务是回归,它主要用于预测数值型数据。分类和回归属于监督学习,之所以称之为监督学习,是因为这类算法必须知道预测什么,即目标变量的信息。

与监督学习相对应的是无监督学习,此时数据没有类别信息,也不会给定目标值。在无监督学习中,将数据集合分成由类似的对象组成的多个类的过程被称为聚类;将寻找描述数据统计值的过程称之为密度估计。此外,无监督学习还可以减少数据特征的维度,以便我们可以使用二维或三维图形更加直观地展示数据信息。

\begin{itemize}
    \item 监督学习:k-近邻算法、线性回归、朴素贝叶斯算法、局部加权线性回归、支持向量机、Ridge 回归、决策树、Lasso 最小回归系数估计
    \item 无监督学习:K-均值、最大期望算法、DBSCAN、Parzen窗设计
\end{itemize}