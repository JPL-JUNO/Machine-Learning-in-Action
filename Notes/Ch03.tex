\chapter{决策树\label{chapter03}}
你是否玩过二十个问题的游戏,游戏的规则很简单:参与游戏的一方在脑海里想某个事物,其他参与者向他提问题,只允许提20个问题,问题的答案也只能用对或错回答。问问题的人通过推断分解,逐步缩小待猜测事物的范围。决策树的工作原理与20个问题类似,用户输入一系列数据,然后给出游戏的答案。我们经常使用决策树处理分类问题,近来的调查表明决策树也是最经常使用的数据挖掘算法\footnote{Giovanni Seni and John Elder, Ensemble Methods in Data Mining: Improving Accuracy Through Combining Predictions, Synthesis Lectures on Data Mining and Knowledge Discovery (Morgan and Claypool, 2010), 28.}。

它之所以如此流行,一个很重要的原因就是不需要了解机器学习的知识,就能搞明白决策树是如何工作的。

k-近邻算法可以完成很多分类任务,但是它最大的缺点就是无法给出数据的内在含义,决策树的主要优势就在于数据形式非常容易理解。

\important{决策树的一个重要任务是为了数据中所蕴含的知识信息,因此决策树可以使用不熟悉的数据集合,并从中提取出一系列规则,在这些机器根据数据集创建规则时,就是机器学习的过程}。
\section{决策树的构造}
\begin{tcolorbox}[title=决策树]
    优点:计算复杂度不高,输出结果易于理解,对中间值的缺失不敏感,可以处理不相关特征数据。

    缺点:可能会产生过度匹配问题。

    适用数据类型:数值型和标称型。
\end{tcolorbox}

首先我们讨论数学上如何使用\textbf{信息论}划分数据集,然后编写代码将理论应用到具体的数据集上,最后编写代码构建决策树。

在构造决策树时,我们需要解决的第一个问题就是,当前数据集上哪个特征在划分数据分类时起决定性作用。为了找到决定性的特征,划分出最好的结果,我们必须评估每个特征。完成测试之后,原始数据集就被划分为几个数据子集。这些数据子集会分布在第一个决策点的所有分支上。如果某个分支下的数据属于同一类型,则无需进一步对数据集进行分割。如果数据子集内的数据不属于同一类型,则需要重复划分数据子集的过程。如何划分数据子集的算法和划分原始数据集的方法相同,直到所有具有相同类型的数据均在一个数据子集内。

创建分支的伪代码函数createBranch()如下所示:
\begin{algorithm}
    \caption{决策树分支判断}
    \eIf{数据集中的每个子项属于同一分类}{
        return 类标签}{
        寻找划分数据集的最好特征\;
        划分数据集\;
        创建分支节点\;
        \For{每个划分的子集}{
            调用函数createBranch并增加返回结果到分支节点中
        }
        return 分支节点(返回子树或者说决策树更好一点)
    }
\end{algorithm}

\section{示例:使用决策树预测隐形眼镜类型}
隐形眼镜数据集\footnote{The dataset is a modified version of the Lenses dataset retrieved from the UCI Machine Learning Repository November 3, 2010 [\url{http://archive.ics.uci.edu/ml/machine-learning-databases/lenses/}]. The source of the data is Jadzia Cendrowska and was originally published in “PRISM: An algorithm for inducing modular rules,” in International Journal of Man-Machine Studies (1987), 27, 349–70.}是非常著名的数据集,它包含很多患者眼部状况的观察条件以及医生推荐的隐形眼镜类型。隐形眼镜类型包括硬材质、软材质以及不适合佩戴隐形眼镜。数据来源于UCI数据库,为了更容易显示数据,这里对数据做了简单的更改。

\figures{Decision tree generated by the ID3 algorithm}
\autoref{Decision tree generated by the ID3 algorithm}所示的决策树非常好地匹配了实验数据,然而这些匹配选项可能太多了。我们将这种问题称之为过度匹配(overfitting)。为了减少过度匹配问题,我们可以裁剪决策树,去掉一些不必要的叶子节点。如果叶子节点只能增加少许信息,则可以删除该节点,将它并入到其他叶子节点中。第9章将进一步讨论这个问题。

\autoref{chapter09}将学习另一个决策树构造算法CART,本章使用的算法称为ID3,它是一个好的算法但并不完美。ID3算法无法直接处理数值型数据,尽管我们可以通过量化的方法将数值型数据转化为标称型数值,但是如果存在太多的特征划分,ID3算法仍然会面临其他问题。
\section{本章小结}
决策树分类器就像带有终止块的流程图,终止块表示分类结果。开始处理数据集时,我们首先需要测量集合中数据的不一致性,也就是熵,然后寻找最优方案划分数据集,直到数据集中的所有数据属于同一分类。ID3算法可以用于划分标称型数据集。构建决策树时,我们通常采用递归的方法将数据集转化为决策树。一般我们并不构造新的数据结构,而是使用Python语言内嵌的数据结构字典存储树节点信息。

使用Matplotlib的注解功能,我们可以将存储的树结构转化为容易理解的图形。Python语言的pickle模块可用于存储决策树的结构。隐形眼镜的例子表明决策树可能会产生过多的数据集划分,从而产生过度匹配数据集的问题。我们可以通过裁剪决策树,合并相邻的无法产生大量信息增益的叶节点,消除过度匹配问题。

还有其他的决策树的构造算法,最流行的是C4.5和CART,\autoref{chapter09}讨论回归问题时将介绍CART算法。

\autoref{chapter02}、\autoref{chapter03}讨论的是结果确定的分类算法,数据实例最终会被明确划分到某个分类中。下一章我们讨论的分类算法将不能完全确定数据实例应该划分到某个分类,或者只能给出数据实例属于给定分类的概率。