\chapter{决策树}
你是否玩过二十个问题的游戏,游戏的规则很简单:参与游戏的一方在脑海里想某个事物,其他参与者向他提问题,只允许提20个问题,问题的答案也只能用对或错回答。问问题的人通过推断分解,逐步缩小待猜测事物的范围。决策树的工作原理与20个问题类似,用户输入一系列数据,然后给出游戏的答案。我们经常使用决策树处理分类问题,近来的调查表明决策树也是最经常使用的数据挖掘算法\footnote{Giovanni Seni and John Elder, Ensemble Methods in Data Mining: Improving Accuracy Through Combining Predictions, Synthesis Lectures on Data Mining and Knowledge Discovery (Morgan and Claypool, 2010), 28.}。

它之所以如此流行,一个很重要的原因就是不需要了解机器学习的知识,就能搞明白决策树是如何工作的。

k-近邻算法可以完成很多分类任务,但是它最大的缺点就是无法给出数据的内在含义,决策树的主要优势就在于数据形式非常容易理解。

\important{决策树的一个重要任务是为了数据中所蕴含的知识信息,因此决策树可以使用不熟悉的数据集合,并从中提取出一系列规则,在这些机器根据数据集创建规则时,就是机器学习的过程}。
\section{决策树的构造}
\begin{tcolorbox}[title=决策树]
    优点:计算复杂度不高,输出结果易于理解,对中间值的缺失不敏感,可以处理不相关特征数据。

    缺点:可能会产生过度匹配问题。

    适用数据类型:数值型和标称型。
\end{tcolorbox}

首先我们讨论数学上如何使用\textbf{信息论}划分数据集,然后编写代码将理论应用到具体的数据集上,最后编写代码构建决策树。

在构造决策树时,我们需要解决的第一个问题就是,当前数据集上哪个特征在划分数据分类时起决定性作用。为了找到决定性的特征,划分出最好的结果,我们必须评估每个特征。完成测试之后,原始数据集就被划分为几个数据子集。这些数据子集会分布在第一个决策点的所有分支上。如果某个分支下的数据属于同一类型,则无需进一步对数据集进行分割。如果数据子集内的数据不属于同一类型,则需要重复划分数据子集的过程。如何划分数据子集的算法和划分原始数据集的方法相同,直到所有具有相同类型的数据均在一个数据子集内。

创建分支的伪代码函数createBranch()如下所示:
\begin{algorithm}
    \caption{决策树分支判断}
    \eIf{数据集中的每个子项属于同一分类}{
        return 类标签}{
        寻找划分数据集的最好特征\;
        划分数据集\;
        创建分支节点\;
        \For{每个划分的子集}{
            调用函数createBranch并增加返回结果到分支节点中
        }
        return 分支节点(返回子树或者说决策树更好一点)
    }
\end{algorithm}